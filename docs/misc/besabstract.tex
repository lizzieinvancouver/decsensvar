\documentclass[11pt]{article}
\usepackage[top=1.00in, bottom=1.0in, left=1in, right=1in]{geometry}
\renewcommand{\baselinestretch}{1.1}
\usepackage{graphicx}
\usepackage{parskip}

\def\labelitemi{--}
\parindent=0pt

\begin{document}
\renewcommand{\refname}{\CHead{}}

EM Wolkovich, J Auerbach and ??

Climate change breaks down a fundamental model of plant biology\\
Subtitle: How warmer days degrade the central limit theory and make a thermal sum model of leafout less predictable

Abstract (196/200 words): \\
Anthropogenic climate change has led to widespread impacts on biological systems, with concerns that increasing warming may fundamentally shift some biological processes, making forecasting and thus mitigation and adaptation more difficult. Plant phenology---the timing of recurring life history events, such as leafout and flowering---has been routinely cited as a potential case of this, with evidence that plants have slowed down their response to warming, become more variable or less predictable often reported. Here, we combine long-term observational data with a random hitting time (or what??) of one of the model fundamental biological models---the thermal sum, or growing degree day model---to understand trends over time. We show that this model predicts declining sensitivity and declining variance with warming, but depends on assumptions that anthropogenic climate change violates. In particular we find that warming climate increases the environmental autocorrelation that plants experience. This outcome---effectively breaking down the central limit theorem for a growing degree day models---predicts increasing variance, which we show in datasets from across the Northern hemisphere. These results show how climate change may fundamentally make biological systems less predictible, with consequences for crops, carbon sequestration and thus climate change itself. \\



Other text: 
Anthrogenic climate change has led to widespread impacts on biological systems, with concerns that increasing warming may force some systems across boundaries or breakpoints. 

\end{document}
